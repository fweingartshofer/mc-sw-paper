\section{Related Work}\label{sec:related-work}
This section will examine the current state of the permission request process of three different permission administrators:
\begin{itemize}
    \item{}Austria's EDA
    \item{}France's Enedis
    \item{}Spain's Datadis
\end{itemize}

The process model described in the subsection\ \ref{subsec:state-process-model} is built using the process models described here as reference.

\subsection{Austria: EDA}\label{subsec:austria:-eda}
The Energy Data Exchange (EDA) is the project and company in austria that manages the permissions to energy consumption, production and master data.
While the first two datatypes are straight forward, master data is data about account details of a metering point.
EDA is owned by all the austrian grid operators.
Unlike the other permission administrators on this list, EDA does not use a pull-based model, but a pushed based approach~\cite{eda}.

A push-based approach for energy consumption and production data has several consequences.
First, if a permission request for energy data is accepted, EDA will send the energy data to the eligible party.
This is done via the AS4 protocol\cite{eda}.

The permission process or the consent process, as it is called in Austria, is based on XML documents that are sent to EDA.
These documents and the process itself are documented at\ \href{https://www.ebutilities.at/}{https://www.ebutilities.at/}.
The figure\ \ref{fig:eda-process-model} shows a rough outline of the austrian permission process~\cite{ebutilities}.

The user initiates the permission process with the intent to share the data with the eligible party.
The eligible party receives data from the user, such as metering point and the name of the users' distribution system operator (DSO).
This allows the eligible party to create a consent management request for energy data or permission request as called in EDDIE as shown in the figure\ \ref{fig:eda-permission-request}.
This permission request is sent to the permission administrator, EDA in Austria, which in turn validates the permission request according to their constraints.
After some specific validations are done, EDA will send an acknowledgment to the eligible party.
EDA further validates the permission request — for example, the start date should be before the end date.
After validation, EDA sends a rejected message in case of validation errors.
Otherwise, the permission request is forwarded to the end user.
The end user can either accept, reject or ignore the permission request.
In the first case, the eligible party will get an accepted message from EDA, for the latter two the permission administrator will send a rejected message~\cite{eda, ebutilities}.

The permission request for austria itself as seen in the figure\ \ref{fig:eda-permission-request} contains a CM Request ID, which the eligible party generates using the eligible party ID and the current timestamp.
The eligible party ID is created by EDA for the eligible party.
Furthermore, the metering point ID or the DSO ID is needed, but one of them can be omitted.
If only the metering point is supplied, the DSO ID can be extracted from the metering point as it contains the DSO ID.
If only the DSO ID is given, the end user has to get the identification for the permission request from the eligible party to accept it in their web portal.
Last, a from and to date is needed; this is the timeframe in which the energy data is requested~\cite{ebutilities}.



\begin{figure}[h]
    \includegraphics[width=\columnwidth]{./assets/eda/cm-process}
    \caption{EDA Process Model}
    \label{fig:eda-process-model}
\end{figure}

\begin{figure}[h]
    \includegraphics[scale=0.3]{./assets/eda/cm-request}
    \caption{EDA Permission Request(CM Request)}
    \label{fig:eda-permission-request}
\end{figure}

\subsection{France: Enedis}\label{subsec:france:-enedis}
France has multiple permission administrators for energy data, one of them is ENEDIS.


\subsection{Spain: Datadis}\label{subsec:spain:-datadis}

\subsection{Comparison of Permission Administrators}\label{subsec:comparison-of-permission-administrators}