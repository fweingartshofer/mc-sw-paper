\section{Related Work}\label{sec:related-work}
This section will examine the current state of the permission request process of three different permission administrators:
\begin{itemize}
    \item{}Austria's EDA
    \item{}France's Enedis
    \item{}Spain's Datadis
\end{itemize}

The process model described in the subsection\ \ref{subsec:state-process-model} is built using the process models described here as reference.

\subsection{Austria: EDA}\label{subsec:austria:-eda}
The Energy Data Exchange (EDA) is the project and company in austria that manages the permissions to energy consumption, production and master data.
While the first two datatypes are straight forward, master data is data about account details of a metering point.
EDA is owned by all the austrian grid operators.
Unlike the other permission administrators on this list, EDA does not use a pull-based model, but a pushed based approach~\cite{eda}.

A push-based approach for energy consumption and production data has several consequences.
First, if a permission request for energy data is accepted, EDA will send the energy data to the eligible party.
This is done via the AS4 protocol\cite{eda}.

The permission process or the consent process, as it is called in Austria, is based on XML documents that are sent to EDA.
These documents and the process itself are documented at\ \href{https://www.ebutilities.at/}{ebutilities.at}.
The figure\ \ref{fig:eda-process-model} shows a rough outline of the austrian permission process~\cite{ebutilities}.

The user initiates the permission process with the intent to share the data with the eligible party.
The eligible party receives data from the user, such as metering point and the name of the users' distribution system operator (DSO).
This allows the eligible party to create a consent management request for energy data or permission request as called in EDDIE as shown in the figure\ \ref{fig:eda-permission-request}.
This permission request is sent to the permission administrator, EDA in Austria, which in turn validates the permission request according to their constraints.
After some specific validations are done, EDA will send an acknowledgment to the eligible party.
EDA further validates the permission request — for example, the start date should be before the end date.
After validation, EDA sends a rejected message in case of validation errors.
Otherwise, the permission request is forwarded to the end user.
The end user can either accept, reject or ignore the permission request.
In the first case, the eligible party will get an accepted message from EDA, for the latter two the permission administrator will send a rejected message.
The last three sections of the sequence diagram show how a permission will end, with either the permission reaching its natural end, the user revoking the permission or the eligible party terminating the permission~\cite{eda, ebutilities}.

The permission request for austria itself as seen in the figure\ \ref{fig:eda-permission-request} contains a CM Request ID, which the eligible party generates using the eligible party ID and the current timestamp.
The eligible party ID is created by EDA for the eligible party.
Furthermore, the metering point ID or the DSO ID is needed, but one of them can be omitted.
If only the metering point is supplied, the DSO ID can be extracted from the metering point as it contains the DSO ID.
If only the DSO ID is given, the end user has to get the identification for the permission request from the eligible party to accept it in their web portal.
Last, a from and to date is needed; this is the timeframe in which the energy data is requested~\cite{ebutilities}.

\begin{figure}[h]
    \includegraphics[width=\columnwidth]{./assets/eda/cm-process-seq}
    \caption{EDA Process Model}
    \label{fig:eda-process-model}
\end{figure}

\begin{figure}[h]
    \includegraphics[scale=0.3]{./assets/eda/cm-request}
    \caption{EDA Permission Request(CM Request)}
    \label{fig:eda-permission-request}
\end{figure}

\subsection{France: Enedis}\label{subsec:france:-enedis}
France has multiple permission administrators for energy data, one of them is ENEDIS.
France utilizes a pull-based model, unlike Austrias' EDA, which is described in subsection\ \ref{subsec:austria:-eda}.
A pull-based approach allows the eligible party to get the data from an interface, like a REST service.

ENEDIS' permission process can be seen in the figure\ \ref{fig:enedis-process-model}.
The process follows the OAuth 2.0 specification, specifically the OAuth 2.0 authorization code grant.
Eligible parties have to register with ENEDIS to get access to a specific set APIs, this is called a ``scope''.
After registering with ENEDIS, the eligible party receives access credentials, a client ID, that need to be presented each API call~\cite{rfc6749-oauth, enedis-dev-guide}.

The OAuth 2.0 authorization code grant enables an end user to provide access to an authorized party.
There are three entities:
\begin{itemize}
    \item{The End User} can authorize others to access their energy data.
    \item{The Authorization Server} can verify the end user and give access to the energy data.
    \item{The Eligible Party} that wants access to and end users' data and is registered with the authorization server.
\end{itemize}
The process begins with the end user initiating it.
Steps 2 to 5 in the figure\ \ref{fig:enedis-process-model} are not required by OAauth 2.0, but are introduced to give the eligible party more control over the permission request.
Here the eligible party has the chance to create, validate the permission request on their end and check for the token endpoint availability of the authorization server.
The end user is then directed to an authorization server, which receives client identification, scope, and redirection URI as parameters.
Subsequently, the authorization server verifies the user and offers them the option to approve the permission request.
In the event of rejection, the authorized party remains uninformed about the refusal.
However, if the request is accepted, the end user is redirected to the specified redirection URI, equipped with an authorization code.
The authorized party hosts this redirection URI and uses the authorization code to get an access token from the authorization server.
This access token, in turn, grants access to the end user's energy data.
If the eligible party wants to terminate a permission request after the permission has been granted, they have to stop using the access tokens, as ENEDIS itself does not offer a way to invalidate credentials.
If the end user decides to revoke permission, the eligible party will only know about this when an API call fails, because the permission has been revoked.
The time limit is saved by the eligible party, so they know when it is reached~\cite{enedis-dev-guide,rfc6749-oauth}.

Since the permission request is mostly done without the eligible partys knowledge, the permission request itself only needs a timeframe in which the eligible party wants to request data.

\begin{figure}[h]
    \includegraphics[width=\columnwidth]{./assets/enedis/process-model-seq}
    \caption{ENEDIS Process Model}
    \label{fig:enedis-process-model}
\end{figure}

\subsection{Spain: Datadis}\label{subsec:spain:-datadis}
Datadis is the permission administrator of Spain.
They also use a pull-based approach to access energy data of an end user.
The permission process is based on HTTP calls, but they do not implement OAuth 2.0~\cite{datadis-dev-guide}.

Datadis' approach to the permission request utilizes JWT as access tokens.
For eligible party to be able to access end user data, they first have to get a token for themselves.
This is done by first registering with Datadis and obtaining a NIF.
A NIF is an identifier for an entity in Datadis' system.
A person's passport number can be their NIF or for a company their registration number.
Once they are registered, they have to obtain the token via an HTTP REST call to an authorization API.
The next step is to create a permission request for a specific end user.
This is done by getting the NIF of the end user, their CUPS, which are the end users metering points, and the requested timeframe, which is essentially a start and end date.
Some of the abbreviations used by Datadis are not documented, such as CUP and NIF.
Once the request is posted to datadis' endpoint, the end user will receive the permission request in their portal and can choose to accept or decline it.
If rejected, the eligible party will not be notified.
If accepted, the eligible party will not be notified.
Therefore, the only solution for the eligible party is to regularly request energy data from the API for an end user account if the result is positive, the permission request was accepted.
This permission process can be seen in the figure\ \ref{fig:datadis-process-model} and the permission request itself in the figure\ \ref{fig:datadis-permission-request}~\cite{datadis-dev-guide, bprt-issues}.

The process if the user decides to revoke a permission, the time limit is reached or the permission is terminated by the eligible party, is equal to ENEDIS' process, as described in subsection\ \ref{subsec:france:-enedis}

\begin{figure}[h]
    \includegraphics[width=\columnwidth]{./assets/datadis/process-model-seq}
    \caption{Datadis Process Model}
    \label{fig:datadis-process-model}
\end{figure}

\begin{figure}[h]
    \includegraphics[scale=0.3]{./assets/datadis/permission-request}
    \caption{Datadis Permission Request}
    \label{fig:datadis-permission-request}
\end{figure}

\subsection{Comparison of Permission Administrators}\label{subsec:comparison-of-permission-administrators}
In this subsection the different approaches between EDA, ENEDIS and Datadis are compared.

EDA implements the approach that is most different to the others.
Their process model requires more steps than the others, as can be seen in the figure\ \ref{fig:eda-process-model}.
EDA also utilizes a different protocol than ENEDIS or Datadis.
While these two use HTTP as the basis, with ENEDIS implementing a standardized OAuth 2.0 authorization code grant, EDAs authorization protocol is based on the AS4 protocol and the Ponton X/P Messenger.

All three all have very different processes, which cannot be mapped one to one.
How all the processes and states can be integrated into one process model can be seen in section\ \ref{sec:state-process-model-design}.
