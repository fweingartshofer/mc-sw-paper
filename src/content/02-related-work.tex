\section{Related Work}\label{sec:related-work}
This section will offer an examination of the current state of the permission request process among three different permission administrators:
\begin{itemize}
    \item{}The Energy Data Exchange (EDA) in Austria
    \item{}France's ENEDIS
    \item{}Spain's Datadis
\end{itemize}

The process model, described in subsection\ \ref{subsec:state-process-model}, is derived from the permission process discussed here.

\subsection{Austria: EDA}\label{subsec:austria:-eda}
The Energy Data Exchange (EDA) refers to the project and company in Austria managing permission requests for energy consumption,
production, and master data.
While the first two data types are self-explanatory, ``master data'' refers to account details for a metering point.
EDA is owned by all Austrian grid operators.
Distinctively, unlike other permission administrators, EDA employs a push-based model rather than a pull-based one~\cite{eda}.

EDAs permission process is the one that has the most states since the eligible party is often notified about changes to the permission request.
The states that EDAs permission process has match up with the permission process model without problems.
Each state has a counterpart in the state process model described in the section\ \ref{sec:state-process-model-design}.

This push-based approach to energy consumption and production data implies several consequences.
Primarily, if a permission request for energy data is approved,
EDA will forward the energy data to the approved party via the AS4 protocol\cite{eda}.
The permission process, referred to as ``consent process'' in Austria, utilizes XML documents to communicate with EDA.
The protocol and documents behind this process are outlined at\ \href{https://www.ebutilities.at/}{ebutilities.at}.
Figure\ \ref{fig:eda-process-model} provides a general outline of the Austrian permission process~\cite{ebutilities}.

The permission process starts when a user expresses intent to share data with an eligible party.
This party receives data from the user, like metering point and the user's Distribution System Operator's (DSO) name.
This information enables the eligible party to initiate a consent management request for energy data,
or permission request as referred to in EDDIE, as demonstrated in Figure\ \ref{fig:eda-permission-request}.
The administrator, in Austria, EDA, then receives this permission request and validates it based on predefined parameters.
Upon successful validation, EDA sends an acknowledgment to the corresponding party.
Further validations ensure the structure and content validity of the request — for instance,
checking whether the start date precedes the end date.
In case of validation failure, EDA sends a rejection message.
Otherwise, the request is relayed to the end user.
The end user has the right to either accept, reject, or ignore the request.
EDA sends an acceptance message to the eligible party in the case of approval and a rejection message for the latter two.
The concluding sections of the sequence diagram illustrate how a permission expires,
with either the permission reaching its natural expiry,
the user revoking the permission, or the eligible party terminating the permission~\cite{eda,ebutilities}.

As demonstrated in Figure\ \ref{fig:eda-permission-request}, the Austrian permission request contains a CM Request ID,
which the eligible party generates with their own ID and the current timestamp.
EDA creates the eligible party ID for the said party.
Additionally, the metering point ID or the DSO ID is needed, although it is possible to omit one of them.
If only the metering point is provided, the DSO ID can be inferred, as it is part of the metering point.
Exclusively providing the DSO ID requires the end user
to find the permission request from the eligible party by its CM Request ID to approve it in their web portal.
Lastly, a start and end date specify the timeframe for the energy data request~\cite{ebutilities}.

\begin{figure}[h]
    \includegraphics[width=\columnwidth]{./assets/eda/cm-process-seq}
    \caption{EDA Process Model}
    \label{fig:eda-process-model}
\end{figure}

\begin{figure}[h]
    \includegraphics[scale=0.3]{./assets/eda/cm-request}
    \caption{EDA Permission Request(CM Request)}
    \label{fig:eda-permission-request}
\end{figure}

\subsection{France: ENEDIS}\label{subsec:france:-enedis}
France fields multiple permission administrators for energy data, one of them being ENEDIS.
Unlike Austria's EDA, France utilizes a pull-based model, as detailed in subsection\ \ref{subsec:austria:-eda}.
ENEDIS employs a pull-based approach that allows the eligible party to retrieve data from an interface,
such as a REST service.

Enedis' permission process does not completely fit the state process model as it does not notify the eligible party about every state change.
This can be fixed by sending the permission request first to the eligible party, which can validate it according to its terms and then send the permission request with the OAuth request to the OAuth server.
That way it is possible to give the eligible party more control over the permission requests.
This will prevent users and eligible parties from creating and accepting permission requests that are not fit for the eligible partys services.

One state that cannot be reached in ENEDIS' permission process for the state process model is the timeout state,
since the REST API will only return the HTTP unauthorized code,
making it impossible to distinguish between timeouts and rejections.

Figure\ \ref{fig:enedis-process-model} shows ENEDIS' permission process,
which adheres to the OAuth 2.0 specification, specifically the authorization code grant scheme.
Eligible parties must first register with ENEDIS to gain access to a specific set of APIs, termed as a ``scope''.
Post-registration (with ENEDIS), the eligible party receives access credentials - a client ID -
that need to be included in each API call~\cite{rfc6749-oauth, enedis-dev-guide}.

The OAuth 2.0 authorization code grant enables an end user to provide access to a validated party.
This process involves three entities:
\begin{itemize}
    \item{The End User}, having the ability to authorize third-party access to their data.
    \item{The Authorization Server}, capable of verifying the end user and providing access to the required data.
    \item{The Eligible Party}, registered with the authorization server, seeking access to the end user's data.
\end{itemize}

The process starts with the end user initiating the consent.
Steps 2 to 5, detailed in Figure\ \ref{fig:enedis-process-model}, while not mandated by OAuth 2.0,
provide the eligible party a greater control over the permission request, an opportunity to validate the request,
and verify the token endpoint availability of the authorization server.
The end user is directed to the authorization server,
receiving client identification, scope, and redirection URI parameters.
Following this, the authorization server verifies the user and prompts the user to approve the permission request.
A rejection remains unnoticed by the approved party.
On approval, the end user is redirected to a specified URI, bearing an authorization code.
Hosting this URI, the authorized party uses the authorization code to obtain an access token from the same server.
This token, in turn, permits access to the requested user's energy data.
If an eligible party wishes to terminate a granted permission request,
they need to cease the use of access tokens, given that ENEDIS does not offer a credential invalidation mechanism.
The eligible party becomes aware of any potential permission revocation by the end user
only if API calls fail due to a change in permissions.
The time constraint is maintained by the eligible party, indicating an expiry~\cite{enedis-dev-guide,rfc6749-oauth}.

As the permission request largely takes part without the knowledge of the eligible party,
the only requisite for the permission request is a specified timeframe for requested data access.

\begin{figure}[h]
    \includegraphics[width=\columnwidth]{./assets/enedis/process-model-seq}
    \caption{ENEDIS Process Model}
    \label{fig:enedis-process-model}
\end{figure}

\subsection{Spain: Datadis}\label{subsec:spain:-datadis}
Datadis serves as Spain's permission administrator.
Similar to France's ENEDIS, they use a pull-based approach to access energy data of an end user.
However, they do not implement OAuth 2.0~\cite{datadis-dev-guide}.

Datadis' permission process is the shortest of all the evaluated permission processes.
It matches the least number of states and does not allow for much eligible party interaction.
The eligible party is never even notified by Datadis that a permission request has been accepted.
A workaround to still get this information is to let the user notify the eligible party by letting them click a button.

Datadis' approach to the permission request process is primarily based on JSON web tokens as API tokens.
For the eligible party to access the end user data, they first have to acquire a token for themselves.
This process begins with the eligible party registering with Datadis and obtaining an identifier or NIF.
Once registered, they must obtain the token via an HTTP REST call to an authorization API.
The subsequent step involves creating a permission request for a specific end user.
The NIF of the end user, their metering points (identified by CUPs),
and the desired timeframe, i.e., start and end dates, form the basis of this request.
Some of the abbreviations utilized by Datadis, such as CUP and NIF, lack clear documentation.
Once posted to Datadis' endpoint, the end user receives the permission request in their portal,
leaving them the choice to accept or decline.
In case a rejection, the eligible party is not notified;
acceptance prompts the user to take further action to inform the eligible party.
This process, as well as the permission request,
can be seen in Figure~\ref{fig:datadis-process-model} and Figure~\ref{fig:datadis-permission-request} respectively~\cite{datadis-dev-guide,bprt-issues}.

When the end user decides to revoke the permission,
when the time limit is reached, or when the permission is terminated by the eligible party,
the process is akin to that of ENEDIS,
as described in subsection~\ref{subsec:france:-enedis}.

\begin{figure}[h]
    \includegraphics[width=\columnwidth]{./assets/datadis/process-model-seq}
    \caption{Datadis Process Model}
    \label{fig:datadis-process-model}
\end{figure}

\begin{figure}[h]
    \centering \includegraphics[scale=0.3]{./assets/datadis/permission-request}
    \caption{Datadis Permission Request}
    \label{fig:datadis-permission-request}
\end{figure}

\subsection{Comparison of Permission Administrators}\label{subsec:comparison-of-permission-administrators}
This subsection aims to present a comparison among the approaches implemented by EDA, ENEDIS, and Datadis.
EDA's approach is significantly distinct from the rest.
Highlighting differences, the process model necessitates more steps than others,
outlined in Figure\ \ref{fig:eda-process-model}.
Further, EDA uses a different protocol compared to ENEDIS or Datadis.
While both ENEDIS and Datadis base their operations on HTTP,
with ENEDIS implementing a standardized OAuth 2.0 authorization code grant,
EDA's authorization protocol is based on the more particular AS4 protocol and the Ponton X/P Messenger.

Despite sharing the common goal,
the three entities possess very distinctive processes which do not directly map onto each other.
The integration of all processes and states into a single process model is demonstrated in section\ \ref{sec:state-process-model-design}.