\section{Conclusion}\label{sec:conclusion}
This section evaluates how the process model implemented in section\ \ref{sec:state-process-model-design} matches the permission process described in section\ \ref{sec:related-work}.
Each permission process will be compared to the state process model.
Furthermore, the skipped or replaced states are explained.

\subsection{Permission State Model in EDA's Permission Process}\label{subsec:permission-state-model-in-eda's-permission-process}
EDAs permission process is the one that has the most states, since the eligible party is often notified about changes to the permission request.
The activity diagram\ \ref{fig:eda-process-model} contains numbers that are matched with the states described in\ \ref{subsec:state-process-model}.
So the following activities are represented by the state next to the number:
\begin{enumerate}
    \item Does not have a corresponding state
    \item Same as above
    \item Created-State
    \item Validated-State
    \item Pending-Permission-Administrator-Acknowledgment-State
\end{enumerate}
