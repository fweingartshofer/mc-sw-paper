\section{Conclusion}\label{sec:conclusion}
This section describes the performance of the implemented process model and outlines future work.

The ``Clean Energy Package'' opened the possibility to access energy data from European Union member states
and necessitated the development of an abstract model
to gain permission to collect energy data from each country's permission administrators.
In the previous sections, the permission processes of three such administrators were examined:
Austria's EDA, France's Enedis, and Spain's Datadis.
Additionally, the specific states and transitions of the permission process model were outlined and justified.

\subsection{Results}\label{subsec:results}
The implementation of the process model shows promising results.
It enhances the management of all permission processes and improves fault tolerance.
For instance, when permission requests fail to be sent to the permission administrator, due to an error,
they can be automatically reattempted.
Initially, this model was only implemented for EDA, Enedis, and Datadis,
and the states of these varied permission processes could be efficiently mapped without major complications.

Implementing the permission processes with the state process model illustrated that some states could be streamlined,
for example, the sent-to-permission-administrator state.
This state is only explicitly present in the EDA
as they issue a notification once the permission request has been acknowledged.
Conversely, for other permission administrators, this state is not present.

\subsection{Future Work}\label{subsec:future-work}
Currently, the process model is implemented for only a few permission processes.
The next phase involves expanding support to include permission administrators from other countries,
such as Denmark's Energinet or Finland's Fingrid.
The primary goal of our model is to integrate with permission processes across all European Union countries.

When a new permission process is implemented with the process model,
it is crucial to reevaluate the defined states and transitions.
This examination ensures that each state is correctly implemented and understood,
and identifies if any additional necessary states are missing.
