\section{Conclusion}\label{sec:conclusion}
This section evaluates how the process model implemented in section\ \ref{sec:state-process-model-design} matches the permission process described in section\ \ref{sec:related-work}.
Each permission process will be compared to the state process model.
Furthermore, the skipped or replaced states are explained.

\subsection{Permission State Model in EDA's Permission Process}\label{subsec:permission-state-model-in-eda's-permission-process}
EDAs permission process is the one that has the most states, since the eligible party is often notified about changes to the permission request.
The table\ \ref{tab:eda-matched-states} shows how each activity in figure\ \ref{fig:eda-process-model} is matched to the states described in subsection\ \ref{subsec:state-process-model}.
The numbers in the left column are the numbers present in the activities in EDAs permission process model.
The table is sorted by the order of the states.

\begin{table}[h]
    \centering
    \begin{tabular}{|c|l|}
        \hline
        \textbf{Activity} & \textbf{State} \\
        \hline
        2 & Created-State \\
        14 & Malformed-State \\
        3 & Validated-State \\
        13 & Unable-to-Send-State \\
        4 & Pending-Permission-Administrator-Acknowledgment-State \\
        5 & Sent-to-Permission-Administrator-State \\
        12 & Invalid-State \\
        10 & Rejected-State \\
        11 & Time-Out-State \\
        8 & Accepted-State \\
        18 & Time-Limit-State \\
        17 & Revoked-State \\
        15 & Terminated-State \\
        \hline
    \end{tabular}
    \caption{Permission Request States matched to EDAs Permission Process}
    \label{tab:eda-matched-states}
\end{table}
