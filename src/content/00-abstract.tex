\begin{abstract}
    EDDIE, the European Distributed Data Infrastructure for Energy, presents a decentralized, open-source Data Space aligned with EU Smart Grids Task Force initiatives.
    It serves as a framework that reduces data integration costs, enabling energy service companies to operate efficiently within a harmonized European market.
    In this context, this paper introduces a central process model aimed at abstracting the diverse data access permission request processes employed by EU member states.
    These processes involve granting customer permission for access to their energy consumption and production data, a crucial part of EDDIE\@.
    The central process model provides a generalized framework that accommodates the differences of individual member state procedures, offering a common reference point for implementation.
    It encompasses various potential scenarios that may occur during the request process.
    With a focus on interoperability, the model not only streamlines data integration but also ensures customer-consent-based access to real-time and historical energy data.

    The state process model tries to solve the problem of having a number of processes,
    which vary in detail, but all follow the same principles.
    The result of this is a set of states and transitions to these states,
    with definitions when a state is transitioned into another one.
\end{abstract}